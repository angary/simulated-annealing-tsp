\documentclass{article}
\usepackage[utf8]{inputenc}
\setlength\parindent{0pt}


\title{
    General NP-Hard Solvers \\
    \large applied to the \\
    Travelling Salesman Problem
}

\author{Gary Sun}

\date{}

\begin{document}

\maketitle

\section{Introduction}
\subsection{NP-Hard Problems}
NP-hard (non-deterministic polynomial-time hard) problems are those that are 
\begin{itemize}
    \item verifiable in polynomial time by a deterministic turing machine OR
    \item solvable in polynomial time by a non-deterministic turing machine
\end{itemize}

Due to their difficulty in solving, there tends to be a general aglo

\subsection{The Travelling Salesman Problem}
The travelling salesman is an example of an NP-hard combinatorial optimisation problem.

It goes as follows:

Given a weighted graph, starting at a vertex, find the minimal cost to travel to all the vertices of the graph once, before returning back to the starting vertex. \\

It has many applications, i.e. in

\begin{itemize}
    \item logistics (determining an optimal route of least cost)
    \item the manufacturing of microchips (determining the path of a soldering iron which has to react all contacts.
\end{itemize}

\section{Deterministic Algorithms}

W

\subsection{Brute Force}

Whilst it is the least efficient algorithm to solve this, it is worth going over its procedure.


\subsection{Branch and Bound}

A branch and bound algorithm is one that explores certain candidate solutions a

\section{Non-Deterministic Algorithms}

\subsection{Genetic Algorithm}

\subsection{Simulated Annealing}

\end{document}
